\documentclass[11pt, letterpaper, titlepage]{article}
\usepackage[utf8]{inputenc}
\usepackage{geometry}
\usepackage{color,graphicx,overpic} 
\usepackage{fancyhdr}
\usepackage{amsmath,amsthm,amsfonts,amssymb}
\usepackage{mathtools}
\usepackage{hyperref}
\usepackage{multicol}
\usepackage{array}
\usepackage{float}
\usepackage{blindtext}
\usepackage{longtable}
\usepackage{scrextend}
\usepackage[font=small,labelfont=bf]{caption}
\usepackage[framemethod=tikz]{mdframed}
\usepackage{calc}
\usepackage{titlesec}
\usepackage{listings}
\usepackage[normalem]{ulem}
\usepackage{tabularx}
\usepackage{mathrsfs}
\usepackage{bookmark}
\usepackage{apple_emoji}
\usepackage{setspace}
\usepackage{ragged2e}
\usepackage{ltablex}
\usepackage{xurl}
\usepackage{siunitx}
\usepackage{lastpage}
\usepackage{enumitem}
\usepackage{minted}
\usepackage{paracol}

\setlist{topsep=0pt}
\mathtoolsset{showonlyrefs}  
\allowdisplaybreaks

\DeclarePairedDelimiter\ceil{\lceil}{\rceil}
\DeclarePairedDelimiter\floor{\lfloor}{\rfloor}
\DeclarePairedDelimiter\bracks{\left(}{\right)}

\newcolumntype{b}{X}
\newcolumntype{s}{>{\hsize=.5\hsize}X}

\definecolor{bg}{rgb}{0.95,0.95,0.95}

\usemintedstyle{tango}
\setminted{linenos}
\setmintedinline{bgcolor=bg,style=bw}

\tikzset{minimum size=1cm}

\geometry{top=2.54cm, left=2.54cm, right=2.54cm, bottom=2.54cm}
\setlength{\headheight}{20pt}
\setlength{\parskip}{0.5cm}
\setlength{\parindent}{0cm}

\newcommand{\B}{\includegraphics[height=1.5em, valign=B, raise=-0.2em]{BigB.png}} 

\pagestyle{fancy}
\fancyhf{}
\rhead{\B enjamin Kong | 1573684}
\lhead{\textit{CMPUT 204 Assignment 6 🚗 🚕 🚙}}
\rfoot{Page \thepage\ of\ \pageref{LastPage}}

\begin{document}
\onehalfspacing

\subsection*{Problem 1.}
\begin{enumerate}[label=\alph*.]
    \item We consider the algorithm presented the lecture slides.
    \begin{enumerate}[label=(\roman*)]
        \item 
        \begin{enumerate}[label=\arabic*.]
            \item \mintinline{text}{power(b, 79)}
            \item \mintinline{text}{power(b, 78)}
            \item \mintinline{text}{power(b, 39)}
            \item \mintinline{text}{power(b, 38)}
            \item \mintinline{text}{power(b, 19)}
            \item \mintinline{text}{power(b, 18)}
            \item \mintinline{text}{power(b, 9)}
            \item \mintinline{text}{power(b, 8)}
            \item \mintinline{text}{power(b, 4)}
            \item \mintinline{text}{power(b, 2)}
            \item \mintinline{text}{power(b, 1)}
            \item \mintinline{text}{power(b, 0)}
        \end{enumerate}

        \item The worst case occurs when each division of $n$ by 2 results in an odd number. When this occurs, each recursive call that divides $n$ by 2 results in another recursive call that subtracts 1 from $n$. So, at most, we need $\ceil*{2\log(n)}$ recursive calls. Note that the $\log(n)$ term appears because when $n$ is even, we divide by 2.
        
        We can derive a worst case by starting from 0, adding 1, then repeating: multiply by 2, add 1 until we get some value greater than 50. We have $0 \rightarrow 1 \rightarrow 2 \rightarrow 3 \rightarrow 6 \rightarrow 7 \rightarrow 14 \rightarrow 15 \rightarrow 30 \rightarrow 31 \rightarrow 62 \rightarrow 63$ (12 total calls). Hence a worst-case scenario occurs when $n = 63$ ($\ceil*{2\log(62)}$ = 12).
    \end{enumerate}

    \item Let $m$ be the smallest power of 2 that is greater than $n$. We can pad the input matrices with zeroes until they become $m \times m$ matrices at which point we can perform matrix multiplication using Strassen's algorithm. This is done by adding $m - n$ zeroes to the end of each row and $m - n$ zeroes to the end of each column (including the new columns formed from adding to each row). We then extract the original matrix after application of Strassen's algorithm by removing the added rows and columns.
    
    For example, let's say our input matrices are $30 \times 30$. In this case, 32 is the smallest power of 2 that is greater than 30. We would pad the input matrices with zeroes such that they become $32 \times 32$ matrices, at which point we apply Strassen's algorithm. This is done by adding two zeroes to the end of each row and two zeroes to the end of each column. After applying Strassen's algorithm, we remove the added rows/columns from the resultant matrix.
\end{enumerate}

\newpage

\subsection*{Problem 2.}
\begin{enumerate}[label=\alph*.]
    \item We consider the algorithm presented in problem 5 of problem set \#4.
    \begin{enumerate}[label=(\roman*)]
        \item The schedule for each day is shown below:
        \begin{tabularx}{\textwidth}{|X|X|X|X|X|X|X|X|}
            \caption{Round-Robin tournament schedule for 8 teams.} \\
            \hline
            Team    & Day 1 & Day 2 & Day 3 & Day 4 & Day 5 & Day 6 & Day 7 \\ \hline
            $t_1$   & $t_2$ & $t_3$ & $t_4$ & $t_5$ & $t_6$ & $t_7$ & $t_8$ \\ \hline
            $t_2$   & $t_1$ & $t_4$ & $t_3$ & $t_6$ & $t_7$ & $t_8$ & $t_5$ \\ \hline
            $t_3$   & $t_4$ & $t_1$ & $t_2$ & $t_7$ & $t_8$ & $t_5$ & $t_6$ \\ \hline
            $t_4$   & $t_3$ & $t_2$ & $t_1$ & $t_8$ & $t_5$ & $t_6$ & $t_7$ \\ \hline
            $t_5$   & $t_6$ & $t_7$ & $t_8$ & $t_1$ & $t_4$ & $t_3$ & $t_2$ \\ \hline
            $t_6$   & $t_5$ & $t_8$ & $t_7$ & $t_2$ & $t_1$ & $t_4$ & $t_3$ \\ \hline
            $t_7$   & $t_8$ & $t_5$ & $t_6$ & $t_3$ & $t_2$ & $t_1$ & $t_4$ \\ \hline
            $t_8$   & $t_7$ & $t_6$ & $t_5$ & $t_4$ & $t_3$ & $t_2$ & $t_1$ \\ \hline
        \end{tabularx}

        \item The running time of the algorithm is given by the recurrence $T(n) = 2T(n/2) + cn^2$ where $c$ is some positive constant.
        
        We apply the master theorem. We have $a = 2$, $b = 2$, and $f(n) = cn^2$. We have that $\log_b(a) = 1$, so $n^{\log_b(a)} = n$. We see that $f(n) \in \Omega(n^{1 + \epsilon})$ for $\epsilon = 1.1$, so case 3 applies. We can also confirm $af(n/b) \leq \delta f(n)$ for some constant $\delta < 1$ for all sufficiently large $n$. We have that 
        \begin{align}
            af(n/b) &= 2c\left(\dfrac{n}{2}\right)^2 \\
            &= \dfrac{cn^2}{2}.
        \end{align}
        Let us select $\delta = 3/4$. Then clearly $af(n/b) \leq \delta f(n)$ for all sufficiently large $n$. As a result, we conclude $T(n) \in \Theta(n^2)$.

        \item The algorithm cannot be generalized for any even $n$ such that it requires at most $n - 1$ days. This is because if $n$ is not a power of 2, then at some point, a recursive call will have an odd number of teams to handle. We provide a counterexample.
        
        Let's consider the case where we have $n = 6$ teams. Our algorithm can have the first half of the teams ($t_1, t_2, t_3$) play the second half of the teams ($t_4, t_5, t_6$) within 3 days:
        \begin{tabularx}{\textwidth}{|X|X|X|X|}
            \caption{Round-Robin tournament schedule for 6 teams (days 3 to 5).} \\
            \hline
            Team  & Day 3 & Day 4 & Day 5 \\ \hline
            $t_1$ & $t_4$ & $t_5$ & $t_6$ \\ \hline
            $t_2$ & $t_5$ & $t_6$ & $t_4$ \\ \hline
            $t_3$ & $t_6$ & $t_4$ & $t_5$ \\ \hline
            $t_4$ & $t_1$ & $t_3$ & $t_2$ \\ \hline
            $t_5$ & $t_2$ & $t_1$ & $t_3$ \\ \hline
            $t_6$ & $t_3$ & $t_2$ & $t_1$ \\ \hline
        \end{tabularx}
        Now that days 3 to 5 are scheduled, in order to satisfy that we don't require more than $n - 1 = 5$ days, we need teams $t_1, t_2, t_3$ to play a tournament between themselves and teams $t_4, t_5, t_6$ to play a tournament between themselves within two days. However, this is impossible for either set of teams: on any given day, at least one team will not be able to play since there are an odd number of teams. This means we need at least 3 days to schedule a tournament between the teams (for example, if $t_1$ doesn't play on the first day, it needs at least 2 more days to play $t_2$ and $t_3$). This shows we require more than $n - 1$ days.
    \end{enumerate}

    \item We consider the algorithm presented in problem 8 of problem set \#4.
    \begin{enumerate}[label=(\roman*)]
        \item \textbf{Cup 3:} We choose $l_3 = n^{2/3}$. We test from step $l_3$, then $2l_3$, then $3l_3$ and so on up until $n / l_3$. If the cup breaks on step $i + 1$, we know the last step at which the cup doesn't break is between $il_3$ and $(i+1)l_3 - 1$. There are at most $n / l_3$ experiments.
        
        \textbf{Cup 2:} We then choose $l_2 = \sqrt{l_3}$. We test from step $il_3 + l_2$, then $il_3 + 2l_2$, then $il_3 + 3l_2$ and so on up until $l_3 / l_2$. If the cup breaks on step $j + 1$, we know the last step at which the cup doesn't break is between $il_3 + jl_2$ and $il_3 + (j+1)l_2 - 1$. There are at most $l_3 / l_2$ experiments. 

        \textbf{Cup 1:} Lastly, we test from step $il_3 + jl_2 + 1$, $il_3 + jl_2 + 2$, $il_3 + jl_2 + 3$, and so on up until $l_2$. Let's say the cup breaks on step $r + 1$. This means the last step at which the cup doesn't break is $il_3 + jl_2 + r$. There are at most $l_2$ experiments.

        \textbf{Analysis:} In the first step, there are at most $n / l_3$ experiments. In the second step, there are at most $l_2 / l_3$ experiments. In the last step, there are at most $l_2$ experiments. Hence, the total number of experiments is given by
        \begin{align}
            T_3(n) &= \dfrac{n}{l_3} + \dfrac{l_3}{l_2} + l_2 \\
            &= \dfrac{n}{n^{2/3}} + \dfrac{n^{2/3}}{\sqrt{l_3}} + \sqrt{l_2} \\
            &= n^{1/3} + \dfrac{n^{2/3}}{\sqrt{n^{2/3}}} + \sqrt{n^{2/3}} \\
            &= n^{1/3} + n^{1/3} + n^{1/3} \\
            &\in O(\sqrt[3]{n}) \\
            &\in o(\sqrt{n}).
        \end{align}

        \item \textbf{Cup 4:} We choose $l_4 = n^{3/4}$. We test from step $l_4$, then $2l_4$, then $3l_4$ and so on up until $n / l_4$. If the cup breaks on step $p + 1$, we know the last step at which the cup doesn't break is between $pl_4$ and $(p+1)l_4 - 1$. There are at most $n / l_4$ experiments. We can then apply the solution described in the previous question. We would simply start with $l_3 = l_4^{2/3}$. We would also start the tests for cup 3 at $pl_4$.
        
        \textbf{Analysis:} The total number of experiments is given by
        \begin{align}
            T_4(n) &= \dfrac{n}{l_4} + \dfrac{l_4}{l_3} + \dfrac{l_3}{l_2} + l_2 \\
            &= \dfrac{n}{n^{3/4}} + \dfrac{n^{3/4}}{(n^{3/4})^{2/3}} + \dfrac{(n^{3/4})^{2/3}}{\sqrt{(n^{3/4})^{2/3}}} + \sqrt{(n^{3/4})^{2/3}} \\
            &= n^{1/4} + \dfrac{n^{3/4}}{n^{2/4}} + \dfrac{n^{2/4}}{n^{1/4}} + n^{1/4} \\
            &= n^{1/4} + n^{1/4} + n^{1/4} + n^{1/4} \\
            &\in O(\sqrt[4]{n}) \\
            &\in o(\sqrt[3]{n}).
        \end{align}
    \end{enumerate}
\end{enumerate}

\newpage

\subsection*{Problem 3.}
We consider the algorithm presented in problem 6 of problem set \#4.
\begin{enumerate}[label=\alph*.]
    \item The algorithm outputs 5. 
    \item The algorithm outputs 5.
    \item The algorithm travels down the array $A$ and sums the numbers it encounters until it finds the first negative number at index $i$. Let us call this sum $S_1$. Once it finds the first negative number, the algorithm finds the sum of the two negative numbers by adding $A[i]$ and $A[i + 1]$ (using the fact that the two negative numbers are consecutive). Let us call this sum $N$. After this, the algorithm starts at index $i + 2$ and sums the remaining elements in the array. Let us call this sum $S_2$. The algorithm then returns the maximum of $S_1$, $S_1 + S_2 + N$, and $S_2$. 
    
    Since the algorithm iterates over each element of the array once, its running time is in $O(n)$. This is faster than the algorithm presented in the problem set: the running time of the algorithm in the problem set is in $O(n\log(n))$.
\end{enumerate}

\newpage

\subsection*{Problem 4.}
We define our input array as $A$ and the value we wish to search for as $v$. Our algorithm returns \mintinline{text}{true} if $v$ is found in $A$ and \mintinline{text}{false} otherwise. Our algorithm begins in the top right corner (i.e. index $i = 1$, $j = m$). We repeat:
\begin{enumerate}
    \item Compare the element in this cell (i.e. $A[i][j]$) with $v$. If it is equal, we return \mintinline{text}{true}.
    \item If the element in this cell is less than $v$ we increment $i$. If $i > n$ then we return \mintinline{text}{false}. Otherwise, we repeat starting from step 1.
    \item Otherwise we decrement $j$. If $j < 1$ then we return \mintinline{text}{false}. Otherwise, we repeat starting from step 1.
\end{enumerate}

The pseudocode is provided below. It is written in Lua. Note that arrays in Lua begin at 1. Furthermore, the \mintinline{text}{local} keyword is used to define a new variable.
\begin{minted}{lua}
function IsInMatrix(A, n, m, v)
    local i = 1
    local j = m
    
    while i <= n and j >= 1 do
        if A[i][j] == v then
            return true
        elseif A[i][j] < v then
            i = i + 1
        else
            j = j - 1
        end
    end
    
    return false
end
\end{minted}

The worst case occurs when the target value is in the bottom left corner. We would need to check $n$ cells in the vertical direction and $m$ cells in the horizontal direction. As a result, we can conclude the running time of the algorithm is in $O(n + m)$.

The best case occurs when the target value is in the top right corner in which case the algorithm returns immediately. This would lead to the running time being in $O(1)$.

\end{document}
