\documentclass[11pt, letterpaper, titlepage]{article}
\usepackage[utf8]{inputenc}
\usepackage{geometry}
\usepackage{color,graphicx,overpic} 
\usepackage{fancyhdr}
\usepackage{amsmath,amsthm,amsfonts,amssymb}
\usepackage{mathtools}
\usepackage{hyperref}
\usepackage{multicol}
\usepackage{array}
\usepackage{float}
\usepackage{blindtext}
\usepackage{longtable}
\usepackage{scrextend}
\usepackage[font=small,labelfont=bf,labelformat=empty]{caption}
\usepackage[framemethod=tikz]{mdframed}
\usepackage{calc}
\usepackage{titlesec}
\usepackage{listings}
\usepackage[normalem]{ulem}
\usepackage{tabularx}
\usepackage{mathrsfs}
\usepackage{bookmark}
\usepackage{apple_emoji}
\usepackage{setspace}
\usepackage{ragged2e}
\usepackage{ltablex}
\usepackage{xurl}
\usepackage{siunitx}
\usepackage{lastpage}
\usepackage{enumitem}
\usepackage{minted}
\usepackage{paracol}

\mathtoolsset{showonlyrefs}  
\allowdisplaybreaks

\DeclarePairedDelimiter\ceil{\lceil}{\rceil}
\DeclarePairedDelimiter\floor{\lfloor}{\rfloor}
\DeclarePairedDelimiter\bracks{\left(}{\right)}

\newcolumntype{b}{X}
\newcolumntype{s}{>{\hsize=.5\hsize}X}

\definecolor{bg}{rgb}{0.95,0.95,0.95}

\usemintedstyle{tango}
\setminted{linenos}
\setmintedinline{bgcolor=bg,style=bw}

\tikzset{minimum size=1cm}

\geometry{top=2.54cm, left=2.54cm, right=2.54cm, bottom=2.54cm}
\setlength{\headheight}{20pt}
\setlength{\parskip}{0.5cm}
\setlength{\parindent}{0cm}

\newcommand{\B}{\includegraphics[height=1.5em, valign=B, raise=-0.2em]{BigB.png}} 

\pagestyle{fancy}
\fancyhf{}
\rhead{\B enjamin Kong | 1573684}
\lhead{\textit{CMPUT 204 Assignment 5 🚗 🚕 🚙}}
\rfoot{Page \thepage\ of\ \pageref{LastPage}}

\begin{document}
\onehalfspacing

\subsection*{Problem 1.}
We consider the algorithm presented in problem 1 of problem set \#4.
\begin{itemize}
    \item If Oomaca can make it to the next location before his bottle is empty, he will not stop. This means he will go as far as possible with each bottle.
    
    \item He stops at
    \begin{enumerate}
        \item Location 1 (0.5),
        \item Location 3 (1.4),
        \item Location 6 (2.2),
        \item Location 7 (3.1),
        \item Location 9 (4.1), and
        \item Location 12 (5.0).
    \end{enumerate}

    \item Let $S = \{ q_1, q_2, \ldots, q_n \}$ be an optimal solution and $S' = \{ x_1, x_2, \ldots, x_{n'} \}$ be a solution generated by our algorithm. We wish to show the first greedy choice $x_1$ in $S'$ can be used to replace one or more choices in $S$ so the altered solution of $S$ is still optimal. 
    
    Since $x_1$ is the furthest hole Oomaca can go with one bottle, $q_1 \leq x_1$. We also know he only needs one bottle to go to $x_1$ so there is no disadvantage to stopping at $x_1$ instead of $q_1$. This implies stopping at $x_1$ instead of $q_1$ doesn't result in $q_2, q_3, \ldots, q_n$ needing to change.
\end{itemize}

\newpage

\subsection*{Problem 2.}
We consider the algorithm presented in problem 2 of problem set \#4.
\begin{itemize}
    \item We would have 3 quarters, 2 dimes, and 3 pennies.
    
    \item Assume $S' = \{ q_{25}', q_{10}', q_{5}', q_{1}' \}$ is an optimal solution and let $S = \{ q_{25}, q_{10}, q_{5}, q_{1} \}$ is our greedy solution. We show our greedy choices can be substituted into $S'$ and still result in an optimal solution. 
    
    $S'$ is an optimal solution because 
    \begin{enumerate}
        \item $q_{1}' < 5$ since a nickel could be used otherwise,
        \item $q_{5}' < 2$ since a dime could be used otherwise, and
        \item $q_{5}' + q_{10}' < 3$ since a quarter could be used otherwise (we either have 3 dimes which can be optimized to 1 nickel and 1 quarter, or we have 2 dimes and a nickel which can be optimized to 1 quarter).
    \end{enumerate}
    The above implies
    \begin{enumerate}
        \item There are at most 4 pennies,
        \item There are at most 9 pennies and nickels, and
        \item There are at most 24 pennies, nickels, and dimes.
    \end{enumerate}
    Since our greedy solution always maximizes $q_{25}$, we know $q_{25}' \leq q_{25}$. In fact, we can say $q_{25}' = q_{25}$: if $q_{25}' < q_{25}$ then $q_{25} - q_{25}' > 0$ meaning dimes, nickels, and pennies are being used instead of $q_{25} - q_{25}'$ quarters which contradicts what makes $S'$ an optimal solution. 

    Similarly, our greedy solution always maximizes $q_{10}$ so $q_{10}' \leq q_{10}$. In fact, we can say $q_{10}' = q_{10}$: if $q_{10}' < q_{10}$ then $q_{10} - q_{10}' > 0$ meaning nickels and pennies are being used instead of $q_{10} - q_{10}'$ dimes which contradicts what makes $S'$ an optimal solution. 

    We also maximize $q_{5}$ so $q_{5}' \leq q_{5}$. In fact, we can say $q_{5}' = q_{5}$: if $q_{5}' < q_{5}$ then $q_{5} - q_{5}' > 0$ meaning pennies are being used instead of $q_{5} - q_{5}'$ nickels which contradicts what makes $S'$ an optimal solution.

    We finally conclude that $q_{1}' = q_{1}$ since $25q_{25}' + 10q_{10}' + 5q_{5}' + q_{1}' = 25q_{25} + 10q_{10} + 5q_{5} + q_{1}$. Thus, our greedy solution finds the optimum number of coins.
    
    \item If the denominations are 0.50, 0.40, and 0.01, then a greedy algorithm for making change for 80 cents would give 1 50¢ coin and 30 pennies for a total of 31 coins. However, the optimal solution is simply 2 40¢ coins for a total of 2 coins.
\end{itemize}

\newpage

\subsection*{Problem 3.}
We consider the algorithm presented in problem 3 of problem set \#4.
\begin{itemize}
    \item Guards are placed at
    \begin{enumerate}
        \item $p_0 = 1.1$ (covers 0.1, 0.5, 1.2, 1.8),
        \item $p_1 = 3.3$ (covers 2.3, 3.1),
        \item $p_2 = 5.5$ (covers 4.5), and
        \item $p_3 = 7.7$ (covers 6.7, 7.5).
    \end{enumerate}

    \item Let $S = \{ p_1, p_2, \ldots, p_k \}$ be an optimal solution (uses the minimum number of guards) and $S' = \{ q_1, q_2, \ldots, q_k \}$ be a solution generated by our algorithm. We need to show that the following two conditions hold:
    \begin{enumerate}
        \item $x_1, \ldots, x_p$ such that $x_p \leq q_1$ must be guarded, and
        \item $p_1 = q_1$ given any problem instance.
    \end{enumerate}
    $x_1$ is the first unguarded painting. Our algorithm places a guard at $q_1 = x_1 + 1$, satisfying the first condition ($x_1, \ldots, x_p$ where $x_p \leq q_1$ will be guarded by placing a guard at $x_1 + 1$). This implies that the first guard's location in an optimal solution $p_1$ cannot be placed to the right of $q_1 = x_1 + 1$ since placing $p_1$ any further to the right means $x_1$ will be unguarded. We can therefore say $p_1 \leq q_1$. We also notice we can replace $p_1$ with $q_1$ since this would not affect the coverage of $x_1$ and would not expose paintings protected by a guard at $q_1$. Therefore we also satisfy the second condition.
\end{itemize}

\newpage

\subsection*{Problem 4.}
An efficient algorithm for computing the optimal order in which to process the customers is to serve the customers in increasing order of $t_i$. Using a sorting algorithm such as quicksort or mergesort, the algorithm would run in $O(n\log(n))$ time. 

Given the problem input $A = [ t_0, t_1, \ldots, t_n ]$, let $B = [ t_{\pi_0}, t_{\pi_1}, \ldots, t_{\pi_n} ]$ such that $t_{\pi_0} \leq t_{\pi_1} \leq \ldots \leq t_{\pi_n}$ be the result of running our algorithm where each $\pi_i$ is the new index after sorting. We now prove the algorithm computes the optimal order via proof by contradiction.

Suppose that our solution $B$ is not optimal. This means there is some solution $C = [ t_{\alpha_0}, t_{\alpha_1}, \ldots, t_{\alpha_n} ]$ that results in a lower total waiting time. Let $j$ be the position at which $B$ and $C$ first differ. Since $t_{\pi_j} \neq t_{\alpha_j}$, we have two cases:
\begin{itemize}
    \item $t_{\pi_j} > t_{\alpha_j}$: this scenario is impossible. Since $B$ is sorted and position $j$ is the first position where $B$ and $C$ differ, any $t_{\alpha}$ smaller than $t_{\pi_j}$ must come before position $j$.
    \item $t_{\pi_j} < t_{\alpha_j}$: solution $C$ results in a higher total waiting time than solution $B$. If $t_{\pi_j} < t_{\alpha_j}$, the total wait time will increase at least $t_{\alpha_j} - t_{\pi_j}$ minutes.
\end{itemize}
We see that solution $C$ cannot be an optimal solution, contradicting the original assumption. So our solution $B$ must be optimal.

\end{document}
